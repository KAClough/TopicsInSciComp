\documentclass[11pt]{article}
\usepackage{graphicx}
\usepackage{booktabs}
\usepackage{makecell}
\usepackage{amsmath}
\usepackage{xcolor}
\usepackage{fullpage}
\usepackage{hyperref}
\linespread{1.25}
\usepackage{listings}
\usepackage{helvet}
\renewcommand{\familydefault}{\sfdefault}
\usepackage[margin=0.75in]{geometry}

\begin{document}
\date{}
\title{\textbf{Topics in Scientific Computing: Week 1 notes} \\ Getting set up}
\maketitle

\section{Schedule}

This week there is no lecture, only a lab session, on Thursday at 2pm. \textbf{I would recommend that you come in person to the lab}, as the two hours should be sufficient to carry out the work below, and you can get help if needed.
However, if you complete the work below in your own time, you do not need to come to the lab session. 
In either case, please make sure you have everything ready before the first lecture on Monday of Week 2. 

\section{Goals of the week}

By the end of the week you should have:
\begin{itemize}
\item{Watched the introduction video.}
\item Got set up with the various bits of software in preparation for the course starting actively next week - more details below. 
\item Filled in the online questionnaire to tell me about any background you have in coding, and what your expectations are for the course.
\end{itemize}

\subsection{Getting set up}

There are several things to do, and they are listed below. If you prefer a more visual introduction, there is a video demonstration on QMPlus, and of course you can come to the lab session for help and advice if you have any problems.

\begin{itemize}
\item Check you have python3
\item Get github
\item Get SageMath
\item Get Latex
\end{itemize}






 




\end{document}